\documentclass[12pt,a4paper,titlepage]{book}
\usepackage[utf8]{inputenc}
\usepackage[german,english]{babel}
\usepackage[T1]{fontenc}
\usepackage{amsmath}
\usepackage{amsfonts}
\usepackage{amssymb}
\usepackage[left=2cm,right=2cm,top=2cm,bottom=2cm]{geometry}
\author{MaMaKow}
\title{Documentation \\pharmacy duty roster}

\begin{document}
\maketitle
\tableofcontents


\chapter{Introduction}
Pharmacy Duty Roster (PDR) is a web application that allows to operate a duty roster for pharmacies.
PDR started in 2015 as an alternative to a really simple excel sheet without formulas.
PDR aims to be user-friendly but at the same time cover all necessary features.
PDR continuously strives to improve. It is open to your requests and wishes.
I hope it will fulfill your expectations.
\section{Getting PDR}
\section{Reporting bugs}
\section{How to contribute}


\chapter{User manual}
\section{The web interface}
You can connect to your PDR instance using any web browser. Just navigate to your server and enter your username and password.
\subsection{Navigation}
By default, the PDR web interface opens to your a menu containing 5 tiles.
You can navigate to:
\begin{itemize}
\item Roster week table view
\item Roster daily view
\item Roster employee view
\item Overtime 
\item Absence
\end{itemize}
\subsubsection{The navigation bar}
In the top there is a navigation bar containing hyperlinks to nearly all the pages of PDR.

\chapter{Administrator manual}
\section{Installation}
\section{Upgrading}
\section{Configuration}
\section{Maintenance}
\section{Issues and Troubleshooting}


\chapter{Developer manual}
\section{Core development}
\section{Documentation}
\section{Testing}
\section{Bug tracker}
\section{Translation}
\subsection{Internationalization}
Different counties have different laws regarding pharmacies and employment. They also have different holidays.
\end{document}