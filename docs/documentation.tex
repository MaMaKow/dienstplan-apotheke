\documentclass[12pt,a4paper,titlepage]{book}
\usepackage[utf8]{inputenc}
\usepackage[german,english]{babel}
\usepackage[T1]{fontenc}
\usepackage{amsmath}
\usepackage{amsfonts}
\usepackage{amssymb}
\usepackage[left=2cm,right=2cm,top=2cm,bottom=2cm]{geometry}
\author{MaMaKow}
\title{Documentation \\pharmacy duty roster}

\begin{document}
\maketitle
\tableofcontents


\chapter{Introduction}
Pharmacy Duty Roster (PDR) is a web application that allows to operate a duty roster for pharmacies.
PDR started in 2015 as an alternative to a really simple excel sheet without formulas.
PDR aims to be user-friendly but at the same time cover all necessary features.
PDR continuously strives to improve. It is open to your requests and wishes.
I hope it will fulfill your expectations.
\section{Getting PDR}
\section{Reporting bugs}
\section{How to contribute}


\chapter{User manual}


\section{The web interface}
You can connect to your PDR instance using any web browser. Just navigate to your server and enter your username and password.

\subsection{Login}
The login page shows the name of the application. You are prompted to enter your username and password.
If you do not have an account yet, you can create one. Just follow the link.
If you have an account, but forgot about your password, or want to change it, you can click on "Lost password".

\subsection{Lost password}
The lost password page shows the name of the application. You are prompted to enter either your username or your email-address at your option. Alternatively you may enter your employee id. 
After you submit the form, an email is sent to your stored email address.
In that email you will find a link. That link will lead you to the password change page.
\subsubsection{Lost password recovery}


\subsection{Create new user account}

\subsection{Navigation}
By default, the PDR web interface opens to your a menu containing 5 tiles.
You can navigate to:
\begin{itemize}
\item Roster week table view
\item Roster daily view
\item Roster employee view
\item Overtime 
\item Absence
\end{itemize}
\subsubsection{The navigation bar}
In the top there is a navigation bar containing hyperlinks to nearly all the pages of PDR.

\subsubsection{Roster week table view}
\subsubsection{Roster daily view}
\subsubsection{Roster employee view}
\subsubsection{Overtime}
\subsubsection{Absence}
There are four views to the absence data.
\begin{itemize}
\item Employee view readonly
\item Employee view edit
\item Monthly table
\item Year overview
\end{itemize}
In the \emph{Employee view readonly} there is a select element to choose the employee to view. There is a button to switch to the edit view.
And there is a table containing the absence data. The columns are start and end of the absence, reason of absence and number of days.
There is a distinct list of possible reasons ( vacation,
        remaining holiday,
       sickness,
        sickness of child,
        unpaid leave of absence,
        paid leave of absence,
        parental leave and
        maternity leave).
The number of days of absence is calculated for a 5 day week. Absences on saturdays and sundays are registered but not counted. The same applys for holidays.


\chapter{Administrator manual}
\section{Installation}
\section{Upgrading}
\section{Configuration}
\section{Maintenance}
\section{Issues and Troubleshooting}


\chapter{Developer manual}
\section{Core development}
\section{Documentation}
\section{Testing}
\section{Bug tracker}
\section{Translation}
\subsection{Internationalization}
Different counties have different laws regarding pharmacies and employment. They also have different holidays.
\end{document}
