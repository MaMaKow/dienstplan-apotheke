% This file was generated with po4a. Translate the source file.
%

\chapter{Entwicklerhandbuch}


\section{Kernentwicklung}
Alle PHP-Skripte laden eine gemeinsame Datei \texttt{default.php}, die die
Standardeinstellungen vornimmt. Sie befindet sich in ./, welches der
\texttt{PDR\_FILE\_SYSTEM\_APPLICATION\_PATH} ist. Siehe folgende Datei:
\lstinputlisting[language=PHP]{../default.php}

\subsection{Ordnerstruktur}
\begin{itemize}
\item \directory{config/} Enthält die Konfigurationsdatei config.php
\item \directory{css/} \emph{veraltet}, verwenden Sie stattdessen
\directory{src/css/}
\item \directory{docs/} Diese Dokumentation und Werkzeuge, um sie zu bauen
\item \directory{img/} Bilder, die vom Programm verwendet werden
\item \directory{js/} \emph{obsolet}, benutzen Sie stattdessen \directory{src/js/}
\item \directory{locale/} Übersetzungsdateien für gettext, derzeit nur Deutsch
(de\_DE)
\item \directory{src/} Der Hauptteil des eigentlichen Quellcodes
    \begin{itemize}
    \item \directory{src/css} Cascading Style Sheets
    \item \directory{src/js} JavasScript
    \item \directory{src/php/} PHP: Hypertext Preprocessor
    \item \directory{src/php/classes/} Enthält alle Klassendateien
class.class\_name.php
    \item \directory{src/php/fragments/} Teile von größeren Seiten, können über PHP
require/include oder mit JavaScript geladen werden
    \item \directory{\textbf{src/php/pages/}} Dies ist der Ort für die einzelnen
Ansichten, welche der menschliche Benutzer verwenden wird, um den Dienstplan
usw. zu sehen.
    \item \directory{src/sql/} SQL-Datenbank Tabellen und Trigger
    \end{itemize}
\item \directory{tests/} Tests, um Fehler im Quellcode zu finden; Dieser Ordner
ist in .gitignore aufgeführt. Nur einige Dateien sind Teil der sichtbaren
Quelle.
\item \directory{tmp/} Ein Verzeichnis für temporäre Dateien. Es gibt noch keine
automatische Bereinigung.
\item \directory{upload/} Das Ziel für hochgeladene Inhalte. Momentan werden nur
spezifische *.PEP-Dateien verstanden, die von Awta ASYS Smart erstellt
wurden. Diese Dateien enthalten Informationen über die Anzahl der Kunden,
die in der Vergangenheit bedient wurden.
\end{itemize}

\subsection{Programmierstil}
Dieses Projekt versucht, einem Programmierstil zu folgen.
\begin{itemize}
\item Bitte vermeiden Sie StudlyCaps und camelCase (Binnenmajuskel).
\item Klassenkonstanten MÜSSEN in Großbuchstaben mit Unterstrichtrennzeichen
deklariert werden.
\item Eigenschaftsnamen MÜSSEN in Unter_strichen geschrieben werden.
\item Einfache Variablen und Objekte werden in Kleinbuchstaben geschrieben.
\item Array-Namen beginnen mit einem einzelnen Großbuchstaben, gefolgt von
Kleinbuchstaben.
\item Methodennamen müssen Unter_strichen geschrieben werden.

\item Der Code MUSS 4 Leerzeichen zum Einrücken verwenden, keine Tabulatoren.
\item Öffnende geschweifte Klammern für Klassen und Funktionen MÜSSEN in derselben
Zeile stehen, und das Schließen geschweifter Klammern MUSS in der nächsten
Zeile nach dem Text beginnen.
\item Öffnende geschweifte Klammern für Kontrollstrukturen SOLLTEN in derselben
Zeile stehen, und das Schließen geschweifter Klammern MUSS in der nächsten
Zeile nach dem Text beginnen.
\end{itemize}

\subsection{Die Datenbank}
Derzeit wird nur MySQL als Datenbankverwaltungssystem (DBMS)
unterstützt. Die Tabellen sind:
\begin{itemize}
\item Abwesenheit (Krankheit, Urlaub und andere Arten von Abwesenheit)
\item approval (saves for each day if the leader has officially authorized the
roster)
\item branch (Informationen über die Hauptapotheke und mögliche Filialen)
\item Dienstplan (die tatsächlichen Dienstplandaten; Anfang, Ende, Pause)
\item employees (Mitarbeiterdaten; Mitarbeiter\_id, Name, Beruf, Fähigkeiten)
\item employees\_backup (eine Kopie der Mitarbeitertabelle mit archivierten
historischen Daten)
\item Feiertage (obsolet)
\item Grundplan\_roll (noch nicht benutzt)
\item Grundplan (the basic plan; Beginning, end, pause; is used to suggest new
rosters)
\item Wartung (obsolet)
\item Mandant (obsolet)
\item Notdienst (Daten der Notdienste und der ihnen zugewiesenen Mitarbeiter)
\item opening\_times\_special (noch nicht benutzt)
\item opening\_times (die Öffnungs- und Schließzeiten der Filialen, noch keine GUI
zur Bearbeitung)
\item pdr\_self (reflects the state of the application itself)
\item pep\_month\_day (the relative amount of work on different days in the month)
\item pep (die rohen Daten zur Arbeitsmenge, gehashed, um die Anzahl der
gelöschten / ignorierten Einträge zu reduzieren)
\item pep\_weekday\_time (die Menge der Arbeit zu verschiedenen Zeiten an
verschiedenen Wochentagen)
\item pep\_year\_month (die relative Menge an Arbeit in verschiedenen Monaten im
Jahr)
\item saturday\_rotation (Wer soll an welchem Samstag arbeiten?)
\item saturday\_rotation\_teams (who belongs to which team for saturday's
rotation?)
\item Schulferien (noch nicht benutzt)
\item Stunden (Überstundenarchiv und Saldo)
\item task\_rotation (rotierende Zuordnung von Mitarbeitern zu einer Aufgabe,
z.B. Rezeptur)
\item user\_email\_notification\_cache (noch nicht benutzt)
\item users\_lost\_password\_token (Token zur Verfügung gestellt, um ein
vergessenes Passwort zu ändern)
\item users\_privileges (die Berechtigungen der Benutzerkonten)
\item users (die Benutzerkonten; Für jeden Benutzeraccount muss genau ein
Mitarbeiter vorhanden sein. Es können Mitarbeiter ohne Benutzerkonten
existieren)
\item Wunschplan (obsolet)
\end{itemize}


Eine Kopie aller Tabellenstrukturen wird in \directory{src/sql/}
gespeichert. Das Verzeichnis enthält auch die Datei
\directory{src/sql/database\_version\_hash.php}, die einen SHA1-Hash aller
Strukturen enthält, die von \verb|SHOW CREATE TABLE|  und \verb|SHOW CREATE
TRIGGER| nach einigen Änderungen zurückgegeben werden. Der Hash wird von
\directory{tests/get-database-structure.php} geschrieben, siehe die Details
in jener Datei.

\subsubsection{Pflege der Datenbank}
Es gibt eine Klasse \emph{update\_database}. Diese Klasse enthält eine
definierte Menge von MySQL-Anweisungen, die die Datenbankstruktur von einem
bekannten Zustand in der Vergangenheit in den aktuellen Zustand versetzen.

Diese Klasse ist nicht gut getestet. Sie könnte funktionieren. Sie könnte
auch die gesamte Datenbank zerstören.

Die Klasse \emph{update\_database} wird bei jeder Anmeldung eines Benutzers
aufgerufen. Sie entscheidet dann selbständig, ob irgendwelche Maßnahmen
ergriffen werden müssen. Um zu entscheiden, wird der in der Datei
\directory{database\_version\_hash.php} gespeicherte Hash mit dem in der
Datenbanktabelle gespeicherten Hash verglichen. \menu{pdr\_self >
pdr\_database\_version\_hash}. \paragraph{Selbstheilende Tabellen} Die
Klasse \emph{database\_wrapper} hat eine Funktion
\emph{create\_table\_from\_template()}, die fehlende Tabellen aus den
Strukturinformationen  in \directory{src/sql/} erstellen kann. Sie wird
aufgerufen, wenn eine PDO-Datenbankabfrage eine Ausnahme mit dem Code 42S02
und dem MySQL-Fehler 1146 auslöst. \section{Dokumentation} Diese
Dokumentation über ein Programm, eine App oder ein Skript ist
unvollständig. Sie können dieses Projekt unterstützen, indem Sie sie
erweitern.

\section{Testen}

\section{Bug Tracker}

\section{Übersetzung}
Lesen Sie diesen Artikel über po4a für die Übersetzung dieses Dokuments:
\href{https://maltris.org/mehrsprachigkeit-fur-fast-alles-po4a-7317.html}{}

\subsection{Internationalisierung}
Different countries have different laws regarding pharmacies and
employment. They also have different holidays.
