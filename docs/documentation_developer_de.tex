\chapter{Entwicklerhandbuch}


\section{Kernentwicklung}
Alle PHP-Skripte laden eine gemeinsame Datei \texttt{default.php}, die die
Standardeinstellungen vornimmt.
Sie befindet sich in \directory{./}, welches der
\texttt{PDR\_FILE\_SYSTEM\_APPLICATION\_PATH} ist.
Siehe folgende Datei:
\lstinputlisting[language=PHP]{../default.php}

\subsection{Ordnerstruktur}
\begin{itemize}
\item \directory{config/} Enthält die Konfigurationsdatei config.php
\item \directory{css/} \emph{veraltet}, verwenden Sie stattdessen
\directory{src/css/}
\item \directory{docs/} Diese Dokumentation und Werkzeuge, um sie zu bauen
\item \directory{img/} Bilder, die vom Programm verwendet werden
\item \directory{js/} \emph{obsolet}, benutzen Sie stattdessen \directory{src/js/}
\item \directory{locale/} Übersetzungsdateien für gettext, derzeit nur Deutsch
(de\_DE)
\item \directory{src/} Der Hauptteil des eigentlichen Quellcodes
    \begin{itemize}
    \item \directory{src/css} Cascading Style Sheets
    \item \directory{src/js} JavasScript
    \item \directory{src/php/} PHP: Hypertext Preprocessor
    \item \directory{src/php/classes/} Enthält alle Klassendateien
class.class\_name.php
    \item \directory{src/php/fragments/} Teile von größeren Seiten, können über PHP
require/include oder mit JavaScript geladen werden
    \item \directory{\textbf{src/php/pages/}} Dies ist der Ort für die einzelnen
Ansichten, welche der menschliche Benutzer verwenden wird, um den Dienstplan
usw. zu sehen.
    \item \directory{src/sql/} SQL-Datenbank Tabellen und Trigger
    \end{itemize}
\item \directory{tests/} Tests, um Fehler im Quellcode zu finden; Dieser Ordner
ist in .gitignore aufgeführt. Nur einige Dateien sind Teil der sichtbaren
Quelle.
\item \directory{tmp/} Ein Verzeichnis für temporäre Dateien. Es gibt noch keine automatische Bereinigung.
\item \directory{upload/} Das Ziel für hochgeladene Inhalte. Momentan werden nur
spezifische *.PEP-Dateien verstanden, die von AwintaOne erstellt wurden.
Diese Dateien enthalten Informationen über die Anzahl der Kunden, die in der Vergangenheit bedient wurden.
\end{itemize}

\subsection{Programmierstil}
Dieses Projekt versucht, einem Programmierstil zu folgen.
\begin{itemize}
\item Klassenkonstanten MÜSSEN in Großbuchstaben mit Unterstrichtrennzeichen
deklariert werden.
\item Einfache Variablen und Objekte werden in Kleinbuchstaben geschrieben.
\item Array-Namen beginnen mit einem einzelnen Großbuchstaben, gefolgt von
Kleinbuchstaben.
\item Der Code MUSS 4 Leerzeichen zum Einrücken verwenden, keine Tabulatoren.
\item Öffnende geschweifte Klammern für Klassen und Funktionen MÜSSEN in derselben
Zeile stehen, und das Schließen geschweifter Klammern MUSS in der nächsten
Zeile nach dem Text beginnen.
\item Öffnende geschweifte Klammern für Kontrollstrukturen SOLLTEN in derselben
Zeile stehen, und das Schließen geschweifter Klammern MUSS in der nächsten
Zeile nach dem Text beginnen.
\end{itemize}
Festplattenplatz ist nicht mehr knapp. IDEs helfen mit Autovervollständigung. Es ist nicht nötig, Dinge abzukürzen.
Bitte verwenden Sie lange Begriffe wie \lstinline|user_email_notification_cache| statt
\lstinline|usr_ml_ntfcn_ca| oder \lstinline|u_e_n_c|.

\subsection{Die Datenbank}
Derzeit wird nur MySQL als Datenbankverwaltungssystem (DBMS)
unterstützt. Die Tabellen sind:
\begin{itemize}
\item absence (Krankheit, Urlaub und andere Arten von Abwesenheit)
\item approval (speichert für jeden Tag, ob der Leiter offiziell den Dienstplan
autorisiert hat)
\item branch (Informationen über die Hauptapotheke und mögliche Filialen)
\item Dienstplan (die tatsächlichen Dienstplandaten; Anfang, Ende, Pause)
\item employees (Mitarbeiterdaten; Mitarbeiter\_id, Name, Beruf, Fähigkeiten)
\item employees\_backup (eine Kopie der Mitarbeitertabelle mit archivierten
historischen Daten)
\item Feiertage (obsolet)
\item principle\_roster (der Grundplan; Anfang, Ende, Pause; wird verwendet, um
neue Dienstpläne vorzuschlagen)
\item Wartung (obsolet)
\item Mandant (obsolet)
\item Notdienst (Daten der Notdienste und der ihnen zugewiesenen Mitarbeiter)
\item opening\_times\_special (noch nicht benutzt)
\item opening\_times (die Öffnungs- und Schließzeiten der Filialen, noch keine GUI
zur Bearbeitung)
\item pdr\_self (spiegelt den Zustand der Anwendung selbst wider)
\item pep\_month\_day (die relative Menge an Arbeit an verschiedenen Tagen im
Monat)
\item pep (die rohen Daten zur Arbeitsmenge, gehashed, um die Anzahl der
gelöschten / ignorierten Einträge zu reduzieren)
\item pep\_weekday\_time (die Menge der Arbeit zu verschiedenen Zeiten an
verschiedenen Wochentagen)
\item pep\_year\_month (die relative Menge an Arbeit in verschiedenen Monaten im
Jahr)
\item saturday\_rotation (Wer soll an welchem Samstag arbeiten?)
\item saturday\_rotation\_teams (Wer gehört zu welcher Mannschaft für die Rotation
am Samstag?)
\item Schulferien (noch nicht benutzt)
\item Stunden (Überstundenarchiv und Saldo)
\item task\_rotation (rotierende Zuordnung von Mitarbeitern zu einer Aufgabe,
z.B. Rezeptur)
\item user\_email\_notification\_cache (noch nicht benutzt)
\item users\_lost\_password\_token (Token zur Verfügung gestellt, um ein
vergessenes Passwort zu ändern)
\item users\_privileges (die Berechtigungen der Benutzerkonten)
\item users (die Benutzerkonten; Für jeden Benutzeraccount muss genau ein
Mitarbeiter vorhanden sein. Es können Mitarbeiter ohne Benutzerkonten
existieren)
\item Wunschplan (obsolet)
\end{itemize}


Eine Kopie aller Tabellenstrukturen wird in \directory{src/sql/}
gespeichert. Das Verzeichnis enthält auch die Datei
\directory{src/sql/database\_version\_hash.php}, die einen SHA1-Hash aller
Strukturen enthält, die von \lstinline|SHOW CREATE TABLE|  und
\lstinline|SHOW CREATE TRIGGER| nach einigen Änderungen zurückgegeben
werden. Der Hash wird von \directory{tests/get-database-structure.php}
geschrieben, siehe die Details in jener Datei.

\subsubsection{Pflege der Datenbank}
Es gibt eine Klasse \emph{update\_database}. Diese Klasse enthält eine
definierte Menge von MySQL-Anweisungen, die die Datenbankstruktur von einem
bekannten Zustand in der Vergangenheit in den aktuellen Zustand versetzen.

Diese Klasse ist nicht gut getestet. Sie könnte funktionieren. Sie könnte
auch die gesamte Datenbank zerstören.

Die Klasse \emph{update\_database} wird bei jedem Login eines Nutzers
aufgerufen. Sie entscheidet dann selbst, ob irgendwelche Maßnahmen ergriffen
werden müssen.  Um dies zu entscheiden, wird der in der Datei
\directory{database\_version\_hash.php} gespeicherte Hash mit dem in der
Datenbanktabelle gespeicherten Hash verglichen
\menu{pdr\_self>pdr\_database\_version\_hash}.

\paragraph{Selbstheilende Tabellen}
Die Klasse \emph{database\_wrapper} hat eine Funktion
\emph{create\_table\_from\_template()}, die in der Lage ist, fehlende
Tabellen aus den Strukturinformationen zu erstellen, die im Verzeichnis
\directory{src/sql/} gespeichert sind. Sie wird aufgerufen, wenn eine
PDO-Datenbankabfrage eine Ausnahme mit dem Code 42S02 und dem MySQL-Fehler
1146 wirft.

\subsection{Klassen}
Die Klassen werden im Ordner \directory{src/php/classes/} oder
\directory{src/php/3rdparty/} gespeichert. Der PHP Autoloader lädt die Klassen
aus \directory{\lstinline|src/php/classes/|}. 
\\Die alte Notation war \directory{\lstinline|class.\$class_name.php|}.
Inzwischen werden auch namespaces unterstützt. 

\subsubsection{DateTime}
\emph{Warum werden die DateTime Objecte überall als clone an die Funktionen übergeben?}
Weil Objekte stets als Referenz übergeben werden.
Änderungen des Datums innerhalb der Methode ändern damit auch das Datumsobjekt außerhalb der Methode.


\subsubsection{Benutzer}
Die Klasse "user" repräsentiert eine Person, die ein Benutzerkonto in PDR registriert hat. Diese Person kann eine Angestellte sein, aber es sind auch Benutzerkonten ohne Verbindung zu Angestellten möglich.

\subsubsection{user\_input}
\paragraph{\lstinline|user_input::get_variable_from_any_input|}
Diese Funktion liest die Benutzereingaben von POST, GET oder COOKIE in
dieser Reihenfolge. Wenn die angeforderten Informationen in einer der Quellen
gefunden werden, werden die anderen ignoriert. Aus Sicherheitsgründen werden
alle Informationen gefiltert. Standardmäßig wird
\lstinline|FILTER_SANITIZE_FULL_SPECIAL_CHARS| eingesetzt. Jeder andere Filter kann als
zweiter Parameter angegeben werden. Wenn in keiner der Quellen Informationen
gefunden werden, wird ein Standardwert (der dritte Parameter) zurückgegeben.



\subsection{Kalender API}
Das Skript unterscheidet nicht zwischen Informationen, die von POST, GET
oder COOKIE gesendet werden. Siehe
\lstinline|user_input::get_variable_from_any_input| für Details. Der
Parameter \lstinline|date_string| akzeptiert jede Zeichenfolge, die von
DateTime interpretiert werden kann. Siehe
\url{https://secure.php.net/manual/de/datetime.formats.date.php} für
Details.


\section{Dokumentation}
Die Dokumentation ist noch nicht vollständig.
Sie können dazu beitragen, dieses Projekt zu erweitern.
Falls Sie Fragen haben oder eine ausführlichere Erklärung benötigen, zögern Sie nicht, mir eine E-Mail zu schreiben oder mich auf GitHub zu kontaktieren.

\section{Testen}
Für die Tests verwende ich Selenium. Die Tests sind in JAVA geschrieben.
Maven wird als Build-Management-Tool eingesetzt.
Um die Tests zu starten muss daher im Verzeichnis \directory{\lstinline|./tests/selenium/|} der Befehl \lstinline|mvn test| aufgerufen werden.

\section{Bug Tracker}
Fehler und Probleme werden bei GitHub verfolgt
\url{https://github.com/MaMaKow/dienstplan-apotheke/issues}


\section{Übersetzung}
Übersetzungen werden mit \lstinline|gettext()| verwaltet.
Ich nutze Poedit um die Datei \directory{\lstinline|./locale/de_DE/LC_MESSAGES/messages.po|} zu bearbeiten.
%%Lesen Sie diesen Artikel über po4a für die Übersetzung dieses Dokuments:
%%\url{https://maltris.org/mehrsprachigkeit-fur-fast-alles-po4a-7317.html}

\subsection{Internationalisierung}
Verschiedene Länder haben unterschiedliche Gesetze in Bezug auf Apotheken
und Beschäftigung.
Sie haben auch unterschiedliche Feiertage.
Momentan ist es noch nicht möglich, Länder oder Bundesländer in der Konfiguration zu hinterlegen.
Es wird daher von Mecklenburg/Vorpommern in Deutschland ausgegangen.

