% This file was generated with po4a. Translate the source file.
%
\chapter{Einführung}
Pharmacy Duty Roster (PDR) ist eine Webanwendung, die es ermöglicht, einen
Dienstplan für Apotheken zu führen. PDR startete 2015 als Alternative zu
einer wirklich einfachen Excel-Tabelle ohne Formeln. PDR möchte
benutzerfreundlich sein und gleichzeitig alle notwendigen Funktionen
abdecken. PDR ist ständig bestrebt, sich zu verbessern. Es ist offen für
Ihre Anfragen und Wünsche. Ich hoffe, es wird Ihre Erwartungen erfüllen.

\section{PDR finden}
Die neueste Version von PDR ist verfügbar unter
\href{https://github.com/MaMaKow/dienstplan-apotheke/releases}{GitHub}.
GitHub stellt den Quellcode als *.zip-Datei oder *.tar.gz Ball zur
Verfügung. Extrahieren Sie die Dateien in einen Ordner.

Stellen Sie sicher, dass Sie PDR in ein Verzeichnis entpacken, auf das Ihr
Webserver zugreifen kann. PHP und der Webserver müssen Lesezugriff auf alle
Dateien und Ordner haben. Es benötigt auch Schreibzugriff auf die
Unterverzeichnisse upload, tmp und config. Möglicherweise möchten Sie den
Webserver-Benutzer zum Besitzer des Verzeichnisses machen mit z. B .:

\begin{lstlisting}
sudo chown -R www-data:www-data /var/www/html/pdr/
\end{lstlisting}

Sie können das Repository auch mit git klonen:
\begin{lstlisting}
git clone https://github.com/MaMaKow/dienstplan-apotheke.git
\end{lstlisting}
Details finden Sie im Administratorhandbuch!

\section{Lizenz}
PDR ist Open-Source-Software unter der AGPL-Lizenz.
\begin{quote}
	Copyright (C) 2018 Dr. Martin Mandelkow
	
	Dieses Programm ist freie Software: Sie können es unter den Bedingungen der
GNU Affero General Public License, wie von der Free Software Foundation
veröffentlicht, entweder Version 3 der Lizenz oder (nach Ihrer Wahl) einer
späteren Version, weitergeben und / oder modifizieren.
	
	Dieses Programm wird in der Hoffnung verbreitet, dass es nützlich sein wird,
aber OHNE JEGLICHE GARANTIE; ohne auch nur die stillschweigende
Gewährleistung der MARKTGÄNGIGKEIT oder EIGNUNG FÜR EINEN BESTIMMTEN
ZWECK. Weitere Informationen finden Sie in der GNU Affero General Public
License.
	
	Sie sollten eine Kopie der GNU Affero General Public License zusammen mit
diesem Programm erhalten haben. Wenn nicht, siehe
<https://www.gnu.org/licenses/>.
	
\end{quote}
 Weitere Informationen finden Sie in der
\href{https://github.com/MaMaKow/dienstplan-apotheke/releases}{Lizenzdatei}.

\section{Fehler melden}
Der Bugtracker befindet sich derzeit bei GitHub
\url{https://github.com/MaMaKow/dienstplan-apotheke/issues}.  GitHub
benötigt ein Konto, um Fehler oder Feature-Anfragen zu melden. Wenn Sie
keinen Account erstellen wollen, können Sie eine E-Mail an
\href{mailto:pdr-issues@martin-mandelkow.de}{pdr-issues@martin-mandelkow.de}
senden.

\section{Wie Sie einen Beitrag leisten können}
Pull requests sind erwünscht. Wenn Sie Änderungen an PDR vorgenommen haben
und diese der Öffentlichkeit beisteuern möchten, können Sie einen Pull
requests auf GitHub öffnen oder Ihre Änderungen auf andere Weise senden.

Sie könnten auch  \lstinline |git send-email| Und Patches an
\href{mailto:pdr-discuss@martin-mandelkow.de}{pdr-discuss@martin-mandelkow.de}
senden
