\chapter{Administrator manual}

\section{Installation}\label{sec:installation}

\subsection{Getting PDR}
The latest release of PDR is available on \href{https://github.com/MaMaKow/dienstplan-apotheke/releases}{GitHub}

You can also get the latest stable version via git:
\begin{lstlisting}
git clone https://github.com/MaMaKow/dienstplan-apotheke.git
\end{lstlisting}
The master branch is tested to be stable.

\subsection{The installer}

\subsubsection{Introduction}
The first page shows some non-technical information about this program. Click \menu{Next} to move on.

\subsubsection{Welcome}
On the second page some technical background information is given. You are informed about the necessary input data, required for continuing the installation. Available database management systems (currently only MySQL) are listed. Finally, you are informed about the user and password strategy for the database access. Click \menu{Next} again, to continue.

\subsubsection{Requirements}
On the next page the application checks, if all requirements are met. These include a minimum PHP version, some PHP extensions and support for database connections. Also the program needs write access to some of its directories.
If problems are found, then a descriptive error message will be shown. It is not possible to continue, until all issues are solved.
Click \menu{Next} again, to continue.

\subsubsection{Database configuration}
The application now starts to collect configuration data.
\begin{itemize}
\item Database type
\item hostname
\item port (optional)
\item username \\An existing database user. The user MUST have the privilege to create a database. The user SHOULD have the privilege to create a less privileged user.
\item password \\The database password of the user. If a new user could be created, then a new secure random password will be given to the new user.
\item database name
\end{itemize}
Enter the required data and \menu{Submit} it.

\subsubsection{Administrator configuration}
After the database values are set, some information about the administrator is collected:
\begin{itemize}
    \item User name\\ the name used by the administrator to login into the program in the future. 
    \item Last name\\ this name is connected to the employee id.
    \item Employee id\\ this is used to create an employee, who is connected to the administrative user.
    \item Contact email address\\ is used for questions and comments from the users. Also this email will receive some internal information from the roster.
    \item Administrator password\\ the password used by the administrator to login into the program in the future. 
\end{itemize}
Please register the administrator and click \menu{Submit}.
The data will be written to the file \directory{config/config.php}.
For every user, that uses the program, there has be be exactly one employee. 
\subsection{First steps}
After submitting the administrator configuration, you will be forwarded to the login page.
Login with your administrator credentials.

On your first login you will be prompted with the branch management. 
Please create at least one branch. You can reach this page at all times in the menu \menu{Administration > Branch Management}.

The next logical step is to setup some more employees in the \menu{Administration > Human resource management}.

After all the employees are inserted, you can just start to write rosters (\menu{Daily View > Daily input}) or you might create principle rosters for specific weekdays (\menu{Daily view > principle roster daily}) or for distinct employees (\menu{Employee > Principle roster employee}).


\section{Upgrading}
Until now, there is no automatic update mechanism established.
You can regularly download release packages from GitHub. Or you can stay in touch via git:
\begin{lstlisting}
git pull origin master
\end{lstlisting}
\emph{CAVE:} Make sure, that you keep your \directory{config/config.php}! It should not be changed by git, because it is listed in the .gitignore file of this project.


\section{Configuration}
You can manually edit the file \directory{config/config.php}.
The default values are:
\begin{lstlisting}
<?php

$config = array(
        'application_name' => 'PDR',
        'database_management_system' => 'mysql',
        'database_host' => 'localhost',
        'database_name' => '',
        'database_port' => 3306,
        'database_user' => '',
        'database_password' => '',
        'session_secret' => '',
        'error_reporting' => E_ALL,
        'display_errors' => 0,
        'log_errors' => 1,
        'error_log' => PDR_FILE_SYSTEM_APPLICATION_PATH . 'error.log',
        'LC_TIME' => 'C',
        'timezone' => 'Europe/Berlin',
        'language' => 'de_DE',
        'mb_internal_encoding' => 'UTF-8',
        'contact_email' => '',
        'hide_disapproved' => FALSE,
        'email_method' => 'mail',
        'email_smtp_host' => NULL,
        'email_smtp_port' => 587,
        'email_smtp_username' => NULL,
        'email_smtp_password' => NULL,
);
\end{lstlisting}


Never delete the first two lines! If the file does not start with \lstinline|<?php| then PHP will not handle it, meaning that anyone can read its content. 
%\menu{Administration > Human resource management}

Most of these options can also be configured in \menu{Administration > Configuration}
\paragraph{application name}
This name is used in the login page, the page title in the browser and as a subject line in emails sent from the program.
\paragraph{database settings}
\begin{itemize}
	\item database\_management\_system 	Currently only mysql is supported. 	Other possibilities could be: 	PostgreSQL,	Oracle Database, SQLite, Microsoft Access or MongoDB.
	\item database\_host The server running the DBMS. Usually 'localhost', if it is on the same server as the application.
	\item database\_name The name of the database
	\item database\_port For MySQL the standard port is 3306.
	\item database\_user During the installation PDR will try to create the user 'pdr' in the database. In the case of success it will choose a random password and grant all privileges on the pdr database. You can choose any other user with access to the database.
	\item database\_password The database password of the database user.
\end{itemize}
\paragraph{session\_secret}
A secret random string used to define the session name. This is relevant only if multiple instances of PDR are running on the same webserver.
\paragraph{error\_reporting}
Which errors should PHP report? 
\paragraph{display\_errors}
Should errors be directly displayed to the user?
\paragraph{log\_errors}
Should errors be logged in a file?
\paragraph{error\_log}
Where should errors be logged?
\paragraph{LC\_TIME}
In wich language should time strings, such as Monday or January be written? This setting is independant from the 'language' setting.
\paragraph{timezone}
The timezone is necessary to make sense of the raw time data in unix time stamps. (1545038523 = Monday, 17-Dec-18 09:22:03 UTC in RFC~2822)
\paragraph{language}
The language of the terms and messages displayed to the user. Only English and German are supported until now.
\paragraph{mb\_internal\_encoding}
An encoding is a way to tell the computer how bits and bytes are translated into letters (e.g. in UTF8 01000101 means E, 11000011 10100100 means ä). 
\paragraph{contact\_email}
Emails of the users can be sent to the administrator of the pdr instance.
\paragraph{hide\_disapproved}
It is possible, to hide scheduled rosters, until they are approved. By default all rosters are immediately visible to any user.
\paragraph{email settings}
Emails are sent (partly) with the PHPMailer class.
\begin{itemize}
	\item 'email\_method'  PHPMailer supports sending via 'mail', 'sendmail', 'qmail' or 'SMTP'. If SMTP is chosen as the email method, then the SMTP settings will be displayed and have to be filled out:
	\item 'email\_smtp\_host' The address of the SMTP server (e.g. postfix on localhost, gmail-smtp-in.l.google.com for gmail)	
	\item 'email\_smtp\_port' Normally one of 25, 465 or 587 (25 = SMTP, 465 = SMTPS, 587 = STARTTLS)
	\item 'email\_smtp\_username' username of the sending mail account
	\item 'email\_smtp\_password' password of the sending mail account
\end{itemize}



\section{Maintenance}
The file \directory{src/php/background\_maintenance.php} will be called on every login of any user.
It will create an instance of the following classes:
\begin{itemize}
    \item maintenance
    \item update\_database
    %\item auto_upgrader
\end{itemize}

\subsection{class maintenance}
The methods contained in the class maintenance are only called, if the last execution is at least \lstinline|MAINTENANCE_PERIOD_IN_SECONDS| ago.
\lstinline|MAINTENANCE_PERIOD_IN_SECONDS| is set to once a day.

The method \lstinline|user_dialog_email->aggregate_messages_about_changed_roster_to_employees()| is called to send out emails to employees, whose roster has been changed.

The method \lstinline|maintenance->cleanup_overtime()| does not do anything yet. It is meant to be used to clean up overtime of existing employees, that happened before they entered the company.

The method \lstinline|maintenance->cleanup_absence()| does not do anything yet. It is meant to be used to clean up absences of existing employees, that happened before they entered the company.


\subsection{class update\_database}
Before the class update\_database executes any of its methods, it checks if there is any change in the official database structure since the last update.
It compares the pdr\_database\_version\_hash from the table pdr\_self in the database with the value \lstinline|PDR_DATABASE_VERSION_HASH| stored in the file \directory{src/php/database\_version\_hash.php}

The class should contain every SQL Query necessary to change an existing database into the new structure without loosing any data.


\section{Issues and Troubleshooting}
Issues are tracked at GitHub \url{https://github.com/MaMaKow/dienstplan-apotheke/issues}
I am trying to answer any question within 3 days.
