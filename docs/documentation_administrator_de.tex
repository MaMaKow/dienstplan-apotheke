% This file was generated with po4a. Translate the source file.
%
\chapter{Administratorhandbuch}

\section{Installation}\label{sec:installation}

\subsection{PDR finden}
Die neueste Version von PDR ist verfügbar unter
\href{https://github.com/MaMaKow/dienstplan-apotheke/releases}{GitHub}

Sie können auch die neueste stabile Version über Git erhalten:
\begin{lstlisting}
git clone https://github.com/MaMaKow/dienstplan-apotheke.git
\end{lstlisting}
Der master Zweig ist auf Stabilität geprüft.

\subsection{Die Installationsroutine}

\subsubsection{Einführung}
Auf der ersten Seite finden Sie einige nicht-technische Informationen zu
diesem Programm. Klicken Sie auf \menu{Next}, um weiterzugehen.

\subsubsection{Willkommen}
Auf der zweiten Seite sind einige technische Hintergrundinformationen
gegeben. Sie werden über die für die Fortführung der Installation
erforderlichen Eingangsdaten informiert. Die verfügbaren
Datenbankverwaltungssysteme (derzeit nur MySQL) werden
aufgelistet. Schließlich werden Sie über die Nutzer-und Passwortstrategie
für den Datenbankzugriff informiert. Klicken Sie erneut auf \menu{Next}, um
fortzufahren.

\subsubsection{Anforderungen}
Auf der nächsten Seite prüft die Anwendung, ob alle Voraussetzungen erfüllt
sind. Dazu gehören eine minimale PHP-Version, einige PHP-Erweiterungen und
die Unterstützung für Datenbankverbindungen. Das Programm benötigt
Schreibzugriff auf einige seiner Verzeichnisse.  Wenn Probleme gefunden
werden, wird eine beschreibende Fehlermeldung angezeigt. Es ist nicht
möglich, weiterzumachen, bis alle Probleme gelöst sind.  Klicken Sie auf
\menu{Next}, um fortzufahren.

\subsubsection{Datenbankkonfiguration}
Die Anwendung beginnt nun mit der Erhebung von Konfigurationsdaten.
\begin{itemize}
\item Datenbanktyp
\item Hostname
\item Port (optional)
\item username \\Ein bestehender Datenbankbenutzer. Der Benutzer MUSS die
Berechtigung haben, eine Datenbank zu erstellen. Der Benutzer SOLLTE das
Privileg haben, einen weniger privilegierten Benutzer zu erstellen.
\item password \\Das Datenbankpasswort des Benutzers. Wenn ein neuer Benutzer
erstellt werden konnte, wird dem neuen Benutzer ein neues sicheres
zufälliges Passwort gegeben.
\item Name der Datenbank
\end{itemize}
Geben Sie die erforderlichen Daten ein und klicken Sie \menu{Absenden}.

\subsubsection{Administrative Konfiguration}
Nachdem die Datenbankwerte festgelegt wurden, werden einige Informationen
zum Administrator gesammelt:
\begin{itemize}
    \item Benutzername \\ Der Name, mit dem der Administrator sich in Zukunft im
Programm anmeldet. 
    \item Nachname \\ Dieser Name ist mit der Mitarbeiter-ID verbunden.
    \item Mitarbeiter-ID \\ Diese wird verwendet, um einen Mitarbeiter zu erstellen,
der mit dem Benutzer mit Administratorrechten verbunden ist.
    \item Die Kontakt-E-Mail-Adresse \\ wird für Fragen und Kommentare von Benutzern
verwendet. Diese E-Mail erhält auch einige interne Informationen aus dem
Dienstplan.
    \item Administrator-Passwort \\ das Passwort, das der Administrator verwendet, um
sich in Zukunft am Programm anzumelden. 
\end{itemize}
Bitte registrieren Sie den Administrator und klicken Sie auf
\menu{Senden}. Die Daten werden in die Datei \directory{config/config.php}
geschrieben. Für jeden Benutzer, der das Programm benutzt, muss es genau
einen Mitarbeiter geben. \subsection {Erste Schritte} Nach dem Absenden der
Administratorkonfiguration werden Sie auf die Anmeldeseite
weitergeleitet. Melden Sie sich mit Ihren Administrator-Anmeldeinformationen
an.

Bei Ihrer ersten Anmeldung werden Sie mit der Filialverwaltung
konfrontiert. Bitte erstellen Sie mindestens eine Filiale. Sie erreichen
diese Seite jederzeit über das Menü \menu{Administration >
Filialverwaltung}.

Der nächste logische Schritt ist die Einrichtung einiger weiterer
Mitarbeiter in \menu{Administration > Personalverwaltung}.

Nachdem alle Mitarbeiter eingefügt sind, können Sie mit dem Schreiben von
Dienstpläne beginnen (\menu{Tagesansicht > Tagesansicht Eingabe}) oder Sie
können Grundpläne erstellen für bestimmte Wochentage (\menu{ Tagesansicht >
Grundplan Tagesansicht}) oder für unterschiedliche Mitarbeiter
(\menu{Mitarbeiter > Grundplan Mitarbeiter}).


\section{Aktualisierung}
Bis jetzt gibt es keinen automatischen Update-Mechanismus. Sie können
regelmäßig Release-Pakete von GitHub herunterladen. Oder Sie können über Git
in Kontakt bleiben:
\begin{lstlisting}
git pull origin master
\end{lstlisting}
\emph{CAVE:} Stellen Sie sicher, dass Sie Ihr \directory{config/config.php}
behalten! Es sollte nicht von Git geändert werden, da es in der
.gitigore-Datei dieses Projekts aufgeführt ist.


\section{Konfiguration}
Sie können die Datei \directory{config/config.php} manuell bearbeiten.  Die
Standardwerte sind:
\begin{lstlisting}
<?php

$config = array( 'application_name' => 'PDR', 'database_management_system'
=> 'mysql', 'database_host' => 'localhost', 'database_name' => '',
'database_port' => 3306, 'database_user' => '', 'database_password' => '',
'session_secret' => '', 'error_reporting' => E_ALL, 'display_errors' => 0,
'log_errors' => 1, 'error_log' => PDR_FILE_SYSTEM_APPLICATION_PATH
. 'error.log', 'LC_TIME' => 'C', 'timezone' => 'Europe/Berlin', 'language'
=> 'de_DE', 'mb_internal_encoding' => 'UTF-8', 'contact_email' => '',
'hide_disapproved' => FALSE, 'email_method' => 'mail', 'email_smtp_host' =>
NULL, 'email_smtp_port' => 587, 'email_smtp_username' => NULL,
'email_smtp_password' => NULL, );
\end{lstlisting}


Löschen Sie niemals die ersten zwei Zeilen! Wenn die Datei nicht mit
\lstinline|<?php| beginnt, wird PHP sie nicht verarbeiten, was bedeutet,
dass jeder ihren Inhalt lesen kann. 


Die meisten dieser Optionen können auch in \menu{Administration >
Konfiguration} \paragraph{Anwendungsname} konfiguriert werden. Dieser Name
wird auf der Anmeldeseite, dem Seitentitel im Browser und als Betreffzeile
in E-Mails verwendet, die vom Programm gesendet werden.
\paragraph{Datenbank Einstellungen}
\begin{itemize}
	\item database\_management\_system  Derzeit wird nur MySQL unterstützt. Andere
Möglichkeiten könnten sein: PostgreSQL, Oracle Database, SQLite, Microsoft
Access oder MongoDB.
	\item database\_host Der Server, auf dem das DBMS ausgeführt wird. Normalerweise
'localhost', wenn es sich auf demselben Server befindet wie die Anwendung.
	\item database\_name Der Name der Datenbank
	\item database\_port Für MySQL ist der Standardport 3306.
	\item database\_user  Während der Installation versucht PDR, den Benutzer 'pdr' in
der Datenbank zu erstellen. Im Erfolgsfall wählt das Programm ein zufälliges
Passwort und gewährt alle Berechtigungen für die pdr-Datenbank. Sie können
jeden anderen Benutzer auswählen, der Zugriff auf die Datenbank hat.
	\item database\_password Das Datenbankkennwort des Datenbankbenutzers.
\end{itemize}
\paragraph{session\_secret}
Eine geheime zufällige Zeichenfolge, mit der der Sitzungsname definiert
wird. Dies ist nur relevant, wenn mehrere Instanzen von PDR auf demselben
Webserver ausgeführt werden. \paragraph{error\_reporting} Welche Fehler
sollte PHP melden? \paragraph{display\_errors} Sollten dem Benutzer Fehler
direkt angezeigt werden? \paragraph{log\_errors} Sollten Fehler in einer
Datei protokolliert werden?\paragraph{error\_log} Wo sollen Fehler
protokolliert werden? \paragraph{LC\_TIME} In welcher Sprache sollten
Zeitzeichenfolgen wie Montag oder Januar geschrieben werden? Diese
Einstellung ist unabhängig von der Einstellung
'language'.\paragraph{timezone} Die Zeitzone ist notwendig, um die
Rohzeitdaten in Unix-Zeitstempeln zu verstehen. (1545038523 = Montag,
17.-18.12. 09:22:03 UTC in RFC ~ 2822)\paragraph{language} Die Sprache der
Begriffe und Meldungen, die dem Benutzer angezeigt werden. Bisher werden nur
Englisch und Deutsch unterstützt.\paragraph{mb\_internal\_encoding} Eine
Codierung ist eine Möglichkeit, dem Computer mitzuteilen, wie Bits und Bytes
in Buchstaben übersetzt werden (z. B. bedeutet UTF8 01000101 E, 11000011
10100100 ä).\paragraph{contact\email} E-Mails der Benutzer können an den
Administrator der Pdr-Instanz gesendet werden. \paragraph{hide\_disapproved}
Es ist möglich, geplante Dienstpläne auszublenden, bis sie genehmigt
werden. Standardmäßig sind alle Dienstpläne sofort für jeden Benutzer
sichtbar.\paragraph{E-Mail-Einstellungen} E-Mails werden (teilweise) mit der
PHPMailer-Klasse gesendet.
\begin{itemize}
	\item 'email\_method' PHPMailer unterstützt das Versenden per 'mail', 'sendmail',
'qmail' oder 'SMTP'. Wenn Sie SMTP als E-Mail-Methode wählen, werden die
SMTP-Einstellungen angezeigt und müssen ausgefüllt werden:
	\item 'email\_smtp\_host' Die Adresse des SMTP-Servers (z. B. loalhost für lokalen
Postfix, gmail-smtp-in.l.google.com für gmail).	
	\item 'email\_smtp\_port' Normalerweise 25, 465 oder 587 (25 = SMTP, 465 = SMTPS,
587 = STARTTLS)
	\item 'email\_smtp\_username' Benutzername des sendenden E-Mail-Kontos
	\item 'email\_smtp\_password' Passwort des sendenden E-Mail-Kontos
\end{itemize}



\section{Wartung}
Die Datei  \directory{src/php/background\_maintenance.php} wird bei jedem
Login eines beliebigen Benutzers aufgerufen.  Es wird eine Instanz der
folgenden Klassen erzeugen:
\begin{itemize}
    \item Wartung
    \item update\_database
    \end{itemize}

\subsection{Klasse maintenance}
Die Methoden, die in der Klassen maintenance enthalten sind, werden nur dann
aufgerufen, wenn die letzte Ausführung mindestens
\lstinline|MAINTENANCE_PERIOD_IN_SECONDS| her ist.
\lstinline|MAINTENANCE_PERIOD_IN_SECONDS| ist auf einmal am Tag eingestellt.

Die Methode
\lstinline|user_dialog_email->aggregate_messages_about_changed_roster_to_employees()|
wird dazu aufgerufen, E-Mails an Mitarbeiter zu senden, deren Dienstplan
geändert wurde.

Die Methode  \lstinline|maintenance->cleanup_overtime()| tut noch
nichts. Sie soll zum Säubern von Überstunden bestehender Mitarbeiter
eingesetzt werden, welche vor dem Eintritt in das Unternehmen eingetragen
wurden.

Die Methode  \lstinline|maintenance->cleanup_absence()| tut noch nichts. Sie
soll zur Säuberung von Abwesenheiten bestehender Mitarbeiter eingesetzt
werden, welche vor dem Eintritt in das Unternehmen eingetragen wurden.


\subsection{Klasse update\_database}
Bevor die Klasse update\_database eine ihrer Methoden ausführt, prüft sie,
ob sich die offizielle Datenbankstruktur seit der letzten Aktualisierung
geändert hat.  Sie vergleicht den pdr\_database\_version\_hash aus der
Tabelle pdr\_self in der Datenbank mit dem Wert
\lstinline|PDR_DATABASE_VERSION_HASH| In der Datei
\directory{src/php/database\_version\_hash.php}

Die Klasse sollte jede SQL-Abfrage enthalten, die notwendig ist, um eine
bestehende Datenbank in die neue Struktur zu überführen, ohne Daten zu
verlieren.


\section{Probleme und Fehlerbehebung}
Fragen werden bei GitHub
\url{https:/github.com/MaKow/dienstplan-apotheke/issues} bearbeitet. Ich
versuche, jede Frage innerhalb von 3 Tagen zu beantworten.
