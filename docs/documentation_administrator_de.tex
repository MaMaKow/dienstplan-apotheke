% This file was generated with po4a. Translate the source file.
%
\chapter{Administratorhandbuch}
\section{Installation}\label{sec:installation}
\subsection{PDR finden}
Die neueste Version von PDR ist verfügbar unter
\href{https://github.com/MaMaKow/dienstplan-apotheke/releases}{GitHub}

Sie können auch die neueste stabile Version über Git erhalten:
\begin{verbatim}
git clone https://github.com/MaMaKow/dienstplan-apotheke.git
\end{verbatim}
Der Master-Zweig ist auf Stabilität getestet. \subsection{Das
Installationsprogramm} \subsubsection{Einführung} Die erste Seite zeigt
einige nicht-technische Informationen über dieses Programm. Klicken Sie auf
\menu{Weiter}, um fortzufahren. \subsubsection{Willkommen} Auf der zweiten
Seite sind einige technische Hintergrundinformationen angegeben. Sie werden
über die notwendigen Eingabedaten informiert, die für die Fortsetzung der
Installation benötigt werden. Verfügbare Datenbank-Management-Systeme
(derzeit nur MySQL) sind aufgelistet. Abschließend werden Sie über die
Benutzer- und Passwortstrategie für den Datenbankzugriff informiert. Klicken
Sie erneut auf \menu{Weiter}, um fortzufahren. \subsubsection{Anforderungen}
Auf der nächsten Seite prüft die Anwendung, ob alle Voraussetzungen erfüllt
sind. Dazu gehören eine PHP-Mindestversion, einige PHP-Erweiterungen und
Unterstützung für Datenbankverbindungen. Außerdem benötigt das Programm
Schreibzugriff auf einige seiner Verzeichnisse. Wenn Probleme gefunden
werden, wird eine beschreibende Fehlermeldung angezeigt. Es ist nicht
möglich fortzufahren, bis alle Probleme gelöst sind. Klicken Sie erneut auf
\menu{Weiter}, um fortzufahren. \subsubsection{Datenbankkonfiguration} Die
Anwendung beginnt nun, Konfigurationsdaten zu sammeln.
\begin{itemize}
\item Datenbanktyp
\item Hostname
\item Port (optional)
\item username \\Ein bestehender Datenbankbenutzer. Der Benutzer MUSS die
Berechtigung haben, eine Datenbank zu erstellen. Der Benutzer SOLLTE das
Privileg haben, einen weniger privilegierten Benutzer zu erstellen.
\item password \\Das Datenbankpasswort des Benutzers. Wenn ein neuer Benutzer
erstellt werden konnte, wird dem neuen Benutzer ein neues sicheres
zufälliges Passwort gegeben.
\item Name der Datenbank
\end{itemize}
Geben Sie die erforderlichen Daten ein und \menu{Senden} sie. \subsubsection
{Administratorkonfiguration} Nachdem die Datenbankwerte festgelegt wurden,
werden einige Informationen zum Administrator gesammelt:
\begin{itemize}
\item Benutzername \\ Der Name, mit dem der Administrator sich in Zukunft im
Programm anmeldet. 
\item Nachname \\ Dieser Name ist mit der Mitarbeiter-ID verbunden.
\item Mitarbeiter-ID \\ Diese wird verwendet, um einen Mitarbeiter zu erstellen,
der mit dem Benutzer mit Administratorrechten verbunden ist.
\item Die Kontakt-E-Mail-Adresse \\ wird für Fragen und Kommentare von Benutzern
verwendet. Diese E-Mail erhält auch einige interne Informationen aus dem
Dienstplan.
\item Administrator-Passwort \\ das Passwort, das der Administrator verwendet, um
sich in Zukunft am Programm anzumelden. 
\end{itemize}
Bitte registrieren Sie den Administrator und klicken Sie auf
\menu{Senden}. Die Daten werden in die Datei \directory{config/config.php}
geschrieben. Für jeden Benutzer, der das Programm benutzt, muss es genau
einen Mitarbeiter geben. \subsection {Erste Schritte} Nach dem Absenden der
Administratorkonfiguration werden Sie auf die Anmeldeseite
weitergeleitet. Melden Sie sich mit Ihren Administrator-Anmeldeinformationen
an.

Bei Ihrer ersten Anmeldung werden Sie mit der Filialverwaltung
konfrontiert. Bitte erstellen Sie mindestens eine Filiale. Sie erreichen
diese Seite jederzeit über das Menü \menu{Administration >
Filialverwaltung}.

Der nächste logische Schritt ist die Einrichtung einiger weiterer
Mitarbeiter in \menu{Administration > Personalverwaltung}.

Nachdem alle Mitarbeiter eingefügt sind, können Sie mit dem Schreiben von
Dienstpläne beginnen (\menu{Tagesansicht > Tagesansicht Eingabe}) oder Sie
können Grundpläne erstellen für bestimmte Wochentage (\menu{ Tagesansicht >
Grundplan Tagesansicht}) oder für unterschiedliche Mitarbeiter
(\menu{Mitarbeiter > Grundplan Mitarbeiter}).

\section{Aktualisierung}
Bis jetzt gibt es keinen automatischen Update-Mechanismus. Sie können
regelmäßig Release-Pakete von GitHub herunterladen. Oder Sie können über Git
in Kontakt bleiben:
\begin{lstlisting}
git pull origin master
\end{lstlisting}
\emph{CAVE:} Stellen Sie sicher, dass Sie Ihr \directory{config/config.php}
behalten! Es sollte nicht von Git geändert werden, da es in der
.gitigore-Datei dieses Projekts aufgeführt ist. \section{Konfiguration} Sie
können die Datei \directory{config / config.php} manuell bearbeiten. Die
Standardwerte sind:
\begin{verbatim}
'application_name' => 'PDR',
'database_management_system' => 'mysql',
'database_host' => 'localhost',
'database_name' => '',
'database_port' => 3306,
'database_user' => '',
'database_password' => '',
'session_secret' => '',
'error_reporting' => E_ALL,
'display_errors' => 0,
'log_errors' => 1,
'error_log' => PDR_FILE_SYSTEM_APPLICATION_PATH . 'error.log',
'LC_TIME' => 'C',
'timezone' => 'Europe/Berlin',
'language' => 'de_DE',
'mb_internal_encoding' => 'UTF-8',
'contact_email' => '',
'hide_disapproved' => FALSE
\end{verbatim}


Löschen Sie niemals die ersten zwei Zeilen! Wenn die Datei nicht mit "<?php"
beginnt, wird PHP sie nicht verarbeiten, was bedeutet, dass jeder ihren
Inhalt lesen kann. \menu{Administration > Personalverwaltung}
\section{Wartung}
\section{Probleme und Fehlerbehebung}


